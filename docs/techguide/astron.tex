A scheme to calculate the stellar illumination for a given point on a planet's surface is available through the {\it socrates\_illuminate} interface. Once given the orbital parameters and stellar constant this will provide the stellar irradiance, zenith and azimuthal angles for a given time, latitude and longitude.

\section{Calculation of Incoming Insolation}

The following discussion of positional astronomy is abbreviated from  
\cite{Smart44}, where greater detail may be found, particularly
in chapters II, V and VI.  Let 
time $t$ be measured from the beginning of the calendar year and perihelion (the planet's closest approach 
to the star) occur at time $\tau$ .  The mean anomaly $M$ is defined: 
\beq
M=\frac{2 \pi(t-\tau)}{T_{Y}}
\label{CII_eq1} 	
\eeq					
where $T_{Y}$ is the length of the year. The true anomaly $v$, the angular distance of the planet round 
its orbit from perihelion, can be calculated from $M$ for a Keplerian orbit, as an expansion called the equation 
of the centre.  We use the third-order approximation (Smart, pp119-120):
\beq
v=M+(2e-e^{3}/4)\sin M+\frac{5}{4}e^{2} \sin 2M+\frac{13}{12}e^{3} \sin 3M
\label{CII_eq2}
\eeq
where $e$ is the eccentricity of the planet's orbit (a geometric quantity which quantifies how 
non-circular an ellipse is).  The normally directed incoming stellar radiation at the top of the atmosphere is 
given by the inverse-square law:
\beq
S(t)=S_{0}\frac{r_{au}^{2}}{r(t)^{2}}
\label{CII_eq3}
\eeq
where $S_{0}$ is the ``stellar constant'', the incoming stellar radiation on unit area exposed normally at a distance of 1 astronomical unit $r_{au}$, and $r(t)$ is the actual star-planet distance.  From Kepler's Second Law:
\beq
S(t)=S_{0}\bigg( \frac{1 + e \cos v}{a(1 - e^{2})}  \bigg)^{2}
\label{CII_eq3a}
\eeq
where $a$ is the semi-major axis of the planet's orbit in astronomical units.

One other global quantity has to be found, the stellar declination $\delta$, the latitude where the star is 
vertically overhead.  This depends on the planet's rotation, which defines latitude, as well as its position in 
its orbit: 
\beq
\sin \delta = \sin \epsilon \sin \theta
\label{CII_eq4}
\eeq
where $\epsilon$ is the obliquity of the ecliptic (the angle between the axes of the planet's rotation and of its orbit) 
and $\theta$  is the angular distance the planet has travelled round its orbit since the vernal equinox, the point in 
the planet's orbit (reached around 21st March for the Earth) where $\delta$  is zero (the star is overhead at the equator) and 
increasing (moving towards Northern summer).  This is obviously equal to: 
\beq
\sin \delta = \sin \epsilon \sin(v-\lambda)
\label{CII_eq5}
\eeq 					
where $\lambda$ is the angle from perihelion to the vernal equinox (the quantity $\pi-\lambda$  is called the longitude of 
perihelion).\\

For user supplied orbital elements the above calculations are carried out for the middle of the
supplied timestep. Where the fixed or secularly varying orbital elements for the Earth are used the
above calculations are carried out with $t$ for 1200 GMT on the current day. This conveniently 
makes daily mean incoming sunlight a function of latitude only, and introduces negligible error.\\

\subsection{Calculation of stellar zenith angle}

Calculating the incoming radiation perpendicular to the local vertical at some point requires the 
stellar zenith angle $\zeta$ there, i.e. the angle between the star and the local vertical.
At latitude (north) $\phi$, longitude (east) $\lambda$ and time $t$ (now measured from prime meridian stellar 
midnight), geometry gives: 
\beq
\cos \zeta =\cos \phi \cos \delta \cos \Omega + \sin \phi \sin \delta
\label{CII_eq6}
\eeq
where
\beq
\Omega=\lambda+\pi(2t/T_{D}-1)
\label{CII_eq7}
\eeq 
is the hour angle of the star (the angle through which the planet has rotated since local stellar noon) 
and $T_{D}$ is the length of the day (e.g. \cite{Paltridge76}). 

In order to correct for the difference between prime meridian stellar time and universal time (or model time)
we employ the ``Equation of Time''. The varying angular velocity 
of the planet in its orbit causes variation in the length of the stellar day (the time for the star to return to the 
same apparent position in the sky, consisting of one complete rotation of the planet plus as much extra 
rotation as is needed to catch up with the change in the planet/star angle as it goes round the star). For the Earth,
these are of order 10 seconds and accumulate to give differences between solar and mean time of up to 17 
minutes. The {\em Equation of Time} may be calculated from the equations for the planet's orbit.
The original formulation of \cite{Smart44} is generally used for the Earth. A second formulation (\cite{Mueller95})
is available which contains higher order terms and may be necessary for more eccentric orbits.\\ 

The stellar zenith angle is calculated as a mean value over the portion of the timestep that the star is above the
horizon.
Integrating (\ref{CII_eq6}) gives the mean $\cos \zeta$ between hour angles $\Omega_{1}$ and $\Omega_{2}$  : 
\beq
\langle \cos \zeta \rangle =\frac{\cos \phi \cos \delta (\sin \Omega_{2} - \sin \Omega_{1})+ \sin \phi \sin \delta 
    (\Omega_{2}-\Omega_{1})}{\Omega_{2}-\Omega_{1}}
\label{CII_eq8}
\eeq
This must be calculated only for the part of the timestep (if any) when $\cos \zeta$ is positive, i.e. the star 
is up.  Where $\tan \phi \tan \delta >1$  there is perpetual day and where $\tan \phi \tan \delta < -1$  perpetual night. 
At other latitudes the hour angles for sunset ($\Omega_{S}$) and sunrise ($\Omega_{R}$) are: 
\beq
\Omega_{S}=\cos^{-1}(-\tan \phi \tan \delta)
\label{CII_eq8}
\eeq
\beq
\Omega_{R}=-\Omega_{S}
\label{CII_eq9}
\eeq
Thus, for example, at a point where the star sets but does not rise within the timestep, $\cos \zeta$ is given 
by: 
\beq
\langle \cos \zeta \rangle=\frac{\cos \phi \cos \delta (\sin \Omega_{S} - \sin \Omega_{B})}{\Omega_{S}-\Omega_{B}}
                + \sin \phi \sin \delta
\label{CII_eq10}
\eeq 			
where $\Omega_{B}$ is the hour angle for that point at the beginning of the timestep. Two separate periods of daylight 
need to be accounted for if the star sets and then rises during a timestep.

The incoming stellar radiation for the timestep at any location is then $S(t)$ times $\langle \cos \zeta \rangle$ times the fraction of the timestep that the star is up at that point.\\

\subsection{Calculation of stellar azimuth angle}

For certain applications the stellar azimuth angle is also required, such as when dealing with shadows from surface terrain.
Here a mean hour angle ${\overline\Omega}$ is determined for the middle of the period where the star is above the (horizontal) horizon.
If the star sets and rises during the timestep then the mean hour angle is arbitrarily taken as the middle of the timestep.

The solar azimuth angle ($\alpha$) measured clockwise from north, is then calculated using the formulation from \cite{Manners12}:
\beq
\alpha = \atantwo \left( - \cos \delta \sin {\overline\Omega}, ~\sin \delta \cos \phi - \cos \delta \cos {\overline\Omega} \sin \phi \right)
\eeq

\subsection{Specification of orbital elements}

The following parameters for the planet's orbit may be user-defined:

\begin{description}
\item{\bf Epoch} [Julian days]\\
The Julian day number on which the specified orbital elements
apply. The increments are then the change in these orbital
elements per Julian day from the given epoch.

The J2000 epoch is specified by the Julian day number 2451545.0
which equates to midday on the 1st January 2000.

\item{\bf Orbital Eccentricity} [unitless]\\
The orbital eccentricity determines the amount by which the orbit
deviates from a perfect circle. A value of 0 is a circular orbit,
values between 0 and 1 form an elliptic orbit, 1 is a parabolic
escape orbit, and greater than 1 is a hyperbola. (The latter two cases
are not available in this formulation.)

\item{\bf Increment to eccentricity per Julian day from epoch} [/day]\\

\item{\bf Argument of periapsis} [radians]\\
The angle between the ascending node and the periapsis (anticlockwise).

The ascending node in this context is the point in the orbit where the
planet passes through its autumn equinox for the northern hemisphere.
The direction from the star to planet at this point in the orbit is the
zero of the reference frame used here for the orbital elements.
This direction is also where the host star appears in the sky as viewed
from the planet at its northern hemisphere spring equinox. For the Earth
this is known as the First Point of Aries.
(Note, for a planet with zero obliquity, this zero direction is
arbitrarily defined.)

The periapsis is the point of closest approach in the planet's
elliptical orbit. For a circular orbit it is undefined
(can be set to zero.)

As we are using the ascending node as the zero point for the reference
frame here, then by definition, the longitude of the ascending node is
zero, and the argument of periapsis is equal to the longitude of
periapsis. For a planet around the sun this is known as the longitude
of perihelion.

\item{\bf Increment to the argument of periapsis per Julian day from epoch} [radians/day]\\

\item{\bf Obliquity of the orbit} [radians]\\
The angle between the rotational axis of the planet and the orbital axis,
where the direction is defined so the rotation is in the same sense (i.e.
anticlockwise).

\item{\bf Increment to obliquity of the orbit per Julian day from epoch} [radians/day]\\

\item{\bf Semi-major axis} [AU]\\
One half of the major axis of the orbit, equivalent to the average
between the orbital distance at periapsis and apoapsis (perihelion
and aphelion for orbits around the Sun). For the Earth this was
originally defined as 1 AU (astronomical unit), but 1 AU is now
defined as exactly 149597870700 metres.

\item{\bf Increment to semi-major axis per Julian day from epoch} [AU/day]\\

\item{\bf Mean anomaly at epoch} [radians]\\
The fraction of the orbital period since passage through periapsis
expressed as a fraction of a circle in radians. If the orbit were
completely circular then this would describe the actual angular
position of the planet in the orbit (the true anomaly).

This is the mean anomaly at the time of the given epoch.

\item{\bf Increment to mean anomaly per Julian day from epoch} [radians/day]\\
This setting defines the orbital period. It equates to $2\pi$ over the
orbital period in Julian days.

\item{\bf Hour angle at epoch} [radians]\\
The fraction of the planet's day since the host star was over
the planet's prime meridian expressed as a fraction of a circle
in radians. For the Earth, this is the angle the Sun has moved
westwards since solar noon.

This is the hour angle at the time of the given epoch.

\item{\bf Increment to hour angle per Julian day from epoch} [radians/day]\\
This setting defines the mean length of the planet's day. It equates to $2\pi$
times the length of the planet's day over the length of a Julian day.

The actual day length will vary slightly over an eccentric orbit which will
be accounted for by the {\em Equation of Time}.

The mean hour angle increment should be consistent with the planetary angular
rotation rate $\omega$ [radians/second]. If $\omega$ is predefined within a
GCM, then hour\_angle\_inc can be determined as follows:\\
\\
hour\_angle\_inc $=$ $86400\omega$ $-$ mean\_anomaly\_inc $-$ arg\_periapsis\_inc\\
\\
In reality this might not be exactly true if the position of the ascending node
is changing with time (i.e. our reference frame itself is moving).
\end{description}

\subsection{Special cases for length of day}

In the general case the length of the day is defined by setting the hour angle
increment as described above. The following special cases are also available:

\begin{description}
\item{\bf Tidally locked planet}\\
When the hour\_angle\_inc is set to zero the planet will be tidally locked
with one side always facing the star. For this case it is not necessary to
calculate a mean value of the stellar zenith angle and equation~\ref{CII_eq6}
can be used directly.

The longitude at which the star is overhead will still vary with the {\em equation
of time} to allow for an eccentric orbit and a constant rotation rate, and the
stellar declination will vary with the orbital parameters.
\item{\bf Earth day}\\
A preset is available to use an Earth day with hour\_angle\_inc set to exactly
$2\pi$. This avoids any numerical drift due to an inexact user setting of $2\pi$.
\item{\bf Fixed sun}\\
The position of the sun in the sky is fixed to the same value for all points.
The solar zenith and azimuth angles are user supplied.
\end{description}


\subsection{Orbital elements for the Earth}

Presets are available to define the orbital elements of the Earth:

\begin{description}
\item{\bf Earth: Fixed}\\
A standard Earth orbit that does not vary from one year to the next.
The day number of the passage through perihelion is set to a mean
value for the years 1995-2005.

As the orbit is reset at the beginning of each year there will be a
slight discontinuity in the orbital position. The epoch used is
therefore always midday on the first day of the current year.
The orbital elements used are taken from JPL and the Astronomical
Almanac (1984) and equivalent to:\\
\\
  eccentricity = 1.6710222E-02\\
  eccentricity\_inc = 0.0\\
  arg\_periapsis = 1.796767421\\
  arg\_periapsis\_inc = 0.0\\
  obliquity = 0.409092804\\
  obliquity\_inc = 0.0\\
  semimajor\_axis = 1.0\\
  semimajor\_axis\_inc = 0.0\\
  mean\_anomaly = -0.037278428\\
  mean\_anomaly\_inc = 0.01720278179\\
  hour\_angle = 0.0\\
\item{\bf Earth: Secular variation}\\
An Earth orbit with secular variations of the orbital elements
described by \cite{Berger78}.
These use higher order terms than the simple linear increments
available with the user defined orbital elements.
\end{description}

\subsubsection{Option to use a 360 day calendar}

It has been common practice to run Earth climate models with an artifical year of 360 days.
When using this option, the distortions of the seasonal cycle of insolation thus introduced are 
minimised by altering the date of perihelion: 
\beq
\tau_{360}=\tau_{365 \frac{1}{4}}=\frac{360}{365 \frac{1}{4}}+0.71
\label{CII_eq11}
\eeq
where the constant 0.71 is derived from the lengths of the months in the two calendars.
This changes the current mean value of $\tau$ from 2.5 days to 3.2 days for the
{\em Earth:Fixed} preset. The actual value varies from year to year (more because of the
effects of the moon and the other planets on the earth's orbit than because of the variation
in the length of the calendar year) but in the model the only variation is that forecast
mode leap years have a slightly different distribution of insolation from other years.

\subsection{Orbital elements for solar system planets}

A fortran utility is available called {\em orb\_elem} that can generate the 
user defined orbital elements for solar system planets in the required reference
frame of the planet's orbit.

Orbital elements available in the literature are generally given in the Earth's
reference frame so require some coordinate transformation.

The fortran code in {\em src/aux/orb\_elem.f90} should be modified directly in
order to output the elements for a given planet.
