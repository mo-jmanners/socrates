The following programs can be used to generate data for the spectral file. The names of the interactive Fortran programs are listed. Scripts are also available for most programs (beginning with ``C'') to call the respective Fortran routines after reading command line options. Man pages for these scripts (where available) are provided at the end of the chapter.

\begin{description}

\item[{\tt corr\_k}]
This program calculates correlated-$k$ coefficients for prescribed
spectral bands using data from HITRAN .par (line absorption) files,
.xsc (cross-section) files or .cia (collision-induced absorption)
files. It will also calculate water vapour continuum absorption
across the band.

A man page for Ccorr\_k is provided.

\item[{\tt scatter\_90}]
This program calculates the monochromatic single scattering
properties of spherical particles averaged over a size distribution
at a range of specified wavelengths.

A file of wavelengths for the calculation and a file of
refractive indices are read in. A distribution and a
scattering algorithm are supplied. The program calculates
single scattering properties at each of the wavelengths given.

A man page for Cscatter is provided.

\item[{\tt scatter\_average\_90}]
This program reads a file of monochromatic single-scattering
data and averages them across the bands specified in a spectral
file. The averaged values may be written to
a file or fitted using a recognized parametrization.

A man page for Cscatter\_average is provided.

\item[{\tt prep\_spec}]
This program is used to interactively construct a spectral file,
either entirely from scratch or based on an existing file.

For a {\bf new} file the user is asked to supply the basic information
to create a {\em skeleton} spectral file. This includes:

\small{\begin{description}
\item[Block 0: Obtain the summary information:] number of spectral bands,
gases and aerosols present.

\item[Block 1: Obtain the limits on the band wavelengths.]

\item[Block 4: Set the types of absorber in each band:] may be set to zero
at this stage and will automatically be updated when block 5 is
set later.

\item[Block 8: Set the grey continuum types:] these are generally only used
for the self-broadened water vapour continuum, and the foreign-broadened
water vapour continuum when then that is not included with the line data.

\item[Block 14: Exclude regions from bands if required.]

\item[Block 18: Set the types of generalised continua in each band:] this
treatment allows k-terms to be used for the continua and can be used for
the water vapour continua instead of block 8. Collision-induced absorption
continua can also be set.
\end{description}}

Alternatively, an {\bf existing} spectral file is read in.
The user then selects a block of data to add to the file (or modify):

\small{\begin{description}
\item[Block 0: Change number of bands:] a range of bands and their associated
data can be removed from the file.

\item[Block 2: Solar spectrum in each band:] requires a solar spectrum file
to be provided (examples are in \$RAD\_DATA/solar/).

\item[Block 3: Rayleigh scattering in each band:] can use air, hydrogen/helium,
or a custom mixture of gases.

\item[Block 5: k-terms and p, T scaling data:] requires files created by the
{\tt corr\_k} program.

\item[Block 6: Thermal source function in each band:] can use a polynomial fit
(preferred for small temperature ranges) or a look-up table over a given
set of temperatures.

\item[Block 9: Continuum extinction and scaling data:] requires grey continua
(single absorption coefficient per band) data from the {\tt corr\_k} program.

\item[Block 10: Droplet parameters in each band:] requires output from
{\tt scatter\_average}.

\item[Block 11: Aerosol parameters in each band:] requires output from
{\tt scatter\_average}.

\item[Block 12: Ice crystal parameters in each band:] requires output from
{\tt scatter\_average}.

\item[Block 15: Parameters for aerosol optical depths:] requires output from
{\tt scatter\_average}.

\item[Block 17: Spectral variability data in sub-bands:] requires high-resolution
solar spectral variability data files. Formats accepted are currently:

CMIP5 (available from http://solarisheppa.geomar.de/cmip5)

CMIP6 (available from http://solarisheppa.geomar.de/cmip6)

\item[Block 19: Continuum k-terms and T scaling data:] requires files created by the
{\tt corr\_k} program.

\item[Block 20: Photolysis pathways:] requires quantum yield data files and
optionally photoelectron enhancement factors (examples are
in \$RAD\_DIR/examples/sp\_uv/). The standard format is two columns: wavelengths
in nanometres followed by quantum yield for the given reaction. An extended
format is also allowed: if the file starts with *TEMPERATURES, then a list of
temperatures (in Kelvin) will be read followed by column data of wavelengths in
nanometres followed by the quantum yield at each temperature (see
\$RAD\_DIR/examples/sp\_uv/O3\_Matsumi2002\_1.qy).
\end{description}}

This routine is only available in interactive form. There are no command
line options and currently no man page. A number of scripts in the examples
directory pipe data directly to {\tt prep\_spec}. If errors occur it is
recommended to call the script interactively to ensure the data matches the
prompts.

\end{description}

Man pages follow formatted using {\tt man -t}.

\includepdf[pages=-]{Ccorr_k.pdf}
\includepdf[pages=-]{Cscatter.pdf}
\includepdf[pages=-]{Cscatter_average.pdf}
